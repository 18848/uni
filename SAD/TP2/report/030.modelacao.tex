\section*{Modelação e Processamento}
% \subsection*{Relações entre Tabelas (dim_date)}
Após importação das dimensões e tabelas de facto é necessário verificar que todas as informações estão corretamente dispostas. Se for necessário, são feitas alterações à estrutura dos dados para que vá de encontro ao desejado.

\subsection{Relações entre Tabelas}
Visto que as tabelas de facto do \textit{`datamart'} não apresentavam ligações com a dimensão \textit{`dim\_date'}, é necessário estabelecer essas ligações arrastando os campos de cada tabela para o campo \textit{`DATE\_KEY'}.

Campos das tabelas de facto:
\begin{itemize}
    \renewcommand\labelitemi{--}
    \item DueDateKey
    \item OrderDateKey
    \item ShipDateKey
\end{itemize}

\subsection{Hierarquias entre Colunas}
Hierarquias são muito sucintamente a definição de prioridade entre colunas da mesma tabela.
Por exemplo, uma hierarquia para definição de uma data (18/01/2021) seria algo similar a:
\begin{center}
    Dia > Mês > Ano
\end{center}
\hsp Como tal, dada a dimensão \textit{`dim\_date'}, foi definida a seguinte hierarquia:
\begin{center}
    Ano (YEAR\_NUMBER) > Trimestre (QUARTER\_NAME)
\end{center}

Estas hierarquias possibilitam a utilização de `drill-down' ou `drill-up', que simboliza a possibilidade de ver dados mais ou menos detalhados (neste caso ao longo do tempo).

\subsection{Medidas Calculadas}

\begin{table}[H]
    \centering
    \def\arraystretch{1.5}
    
    % \setlength{\tabcolsep}{2em}
    % \renewcommand{\arraystretch}{3}%
    \begin{tabular}{|c|c|c|}
    \hline
         \multicolumn{3}{|c|}{Compras - measure\_purchases}\\
    \hline
        Medida & Descrição & DAX\\
    \hline
        FreightPurchases & \makecell{Total de gastos \\em Transportes} & \makecell{SUMX(\\ DISTINCT(ft\_compras[PurchaseOrderID]), \\FIRSTNONBLANK(ft\_compras[Freight], 0))}\\
    \hline
        PurchaseTotal & \makecell{Total gasto em \\Produtos Comprados} & \makecell{SUM(ft\_compras[LineTotal])}\\
    \hline
        ReceivedQty & \makecell{Total Produtos \\Recebidos} & \makecell{SUM(ft\_compras[ReceivedQty])}\\
    \hline
        RejectedQty & \makecell{Total Produtos \\Rejeitados} & \makecell{SUM(ft\_compras[RejectedQty])}\\
    \hline
        StockedQty & \makecell{Total Produtos \\Armazenados} & \makecell{SUM(ft\_compras[StockedQty])}\\
    \hline
        TaxAmtPurchase & \makecell{Total de Impostos por\\cada Compra} & \makecell{SUMX(\\DISTINCT(ft\_compras[PurchaseOrderID]), \\FIRSTNONBLANK(ft\_compras[TaxAmt], 0))}\\
    \hline
        TotalDuePurchase & \makecell{Total gasto em \\Compras} & \makecell{SUMX(\\DISTINCT ( ft\_compras[PurchaseOrderID] ), \\FIRSTNONBLANK(ft\_compras[TotalDue], 0))}\\
    \hline
    \end{tabular}
    \begin{tabular}{|c|c|c|}
    \hline
         \multicolumn{3}{|c|}{Vendas - measure\_sales}\\
    \hline
        Medida & Descrição & DAX\\
    \hline
        DistinctCustomers & Total de Clientes & \makecell{DISTINCTCOUNTNOBLANK(\\ft\_vendas[CustomerKey])}\\
    \hline
        FreightSales & \makecell{Total de gastos \\em Transportes} & \makecell{SUMX(\\DISTINCT ( ft\_vendas[SalesOrderID] ), \\FIRSTNONBLANK ( ft\_vendas[Freight], 0 ))}\\
    \hline
        QtyProduct & \makecell{Total de Produtos \\Vendidos} & \makecell{SUM(ft\_vendas[OrderQty])}\\
    \hline
        QtySales & \makecell{Total de Vendas} & \makecell{DISTINCTCOUNT(ft\_vendas[SalesOrderID])}\\
    \hline
        SalesTotal & \makecell{Total gerado \\por Vendas} & \makecell{SUM(ft\_vendas[LineTotal]) }\\
    \hline
        TaxAmount & \makecell{Soma dos impostos \\pagos por cada venda.} & \makecell{SUMX(\\DISTINCT ( ft\_vendas[SalesOrderID] ), \\FIRSTNONBLANK ( ft\_vendas[TaxAmt], 0 ))}\\
    \hline
    \end{tabular}
\end{table}