\chapter*{Introdução}
\addcontentsline{toc}{chapter}{Introdução}

A execução do trabalho prático proposto requer a utilização da plataforma Arduino. Foram utilizados, assim como diversos outros componentes compatíveis com a plataforma, de forma a resolver os 4 sistemas e as variadas funções adicionais requeridas.

Para o Sistema de controlo de iluminação interior, o Arduino deverá ser capaz de analisar os valores lidos de um fotorresistor, para assim alterar a intensidade de um LED.

Para o Sistema de controlo de Climatização, o Arduino deverá ser capaz de analisar os valores lidos de um sensor de temperatura e, através da utilização de um display LCD, apresentá-los de forma legível.

Para o Sistema de Acesso ao Estacionamento, o Arduino deverá ser capaz de analisar o input de um utilizador através de um comando de infravermelhos e assim alterar o posicionamento de um motor Servo, sendo que também deverá ser possível suspender o movimento do mesmo.

Para o Sistema de Segurança, o Arduino deverá ser capaz de analisar o input de um sensor de movimento e de seguida acionar um alarme sonoro.

Será ainda necessário, devido às restrições em termos de Hardware, a junção de sistemas por pares para execução em simultâneo das suas funções.

Todos estes sistemas deverão ter em conta as funções adicionais de necessária implementação em pelo menos um dos sistemas. 
