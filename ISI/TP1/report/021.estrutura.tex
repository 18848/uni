\section{Estrutura}

Cada tabela tem as suas operações básicas:
\begin{itemize}
    \item Equipa
    \begin{itemize}
        \item Adição de Novas Equipas
        \item Leitura das Equipas Existentes
        \item Leitura das Equipas com mais valor gasto em requisições
    \end{itemize}
    
    \item Requisição
    \begin{itemize}
        \item Adição de Novas Requisições
        \item Leitura das Requisições Existentes
        \item Alteração do valor "entregue"
    \end{itemize}
    
    \item RequisiçãoMaterial
    \begin{itemize}
        \item Adição de Novas RequisiçõesMaterial
        \item Leitura das RequisiçõesMaterial Existentes
    \end{itemize}
    \item Material
    \begin{itemize}
        \item Adição de Novos Materiais
        \item Leitura dos Materiais Existentes
        \item Leitura dos Materiais mais usados em Requisições
    \end{itemize}
\end{itemize}

Numa tentativa de simplificação acreditamos ser desnecessária as implementações de:
\begin{itemize}
    \item Equipa
    \begin{itemize}
        \item Atualização de Equipa
        \item Remoção de Equipa
    \end{itemize}
    
    \item Requisição
    \begin{itemize}
        \item Remoção de Requisições Existentes
    \end{itemize}
    
    \item RequisiçãoMaterial
    \begin{itemize}
        \item Atualização de RequisiçãoMaterial
        \item Remoção de RequisiçõesMaterial Existentes
    \end{itemize}
    
    \item Material
    \begin{itemize}
        \item Atualização de Materiais
        \item Remoção de Materiais Existentes
    \end{itemize}
\end{itemize}