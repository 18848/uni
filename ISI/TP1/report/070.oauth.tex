\chapter{Conteúdos Extra}

Este último capítulo do trabalho prático abrange os elementos desenvolvidos após a data de entrega original.

\section{OAuth}

Para a implementação de OAuth é utilizada uma API que será utilizada através de uma aplicação cliente em Forms. Esta aplicação cliente foi adaptada para comunicar com a API, pois à data da entrega original foi implementada como uma aplicação offline.

Utilizando JSON Web Tokens, é possível manter uma sessão com autenticação e autorizações aberta. 

Para fazer uso do JWT são encriptados os dados de autenticação assim como informações adicionais (data de criação e tempo de vida, do token).


\section{Documentação}

Foi, para além da secção anterior, adicionada documentação para (algum) código, as funções têm uma breve descrição do seu funcionamento. A documentação pode ser acedida através do ficheiro "documentação" na pasta raíz do projeto.