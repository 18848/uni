\section{Formulação do Problema}
  \subsection*{\bf Descrição}
  
 O problema tem como objetivo encontrar a melhor distribuição de funcionários de um supermercado por um horário definido em turnos de manha e de tarde, tendo em conta o número total de funcionários, assim como as previsões de clientes que visitarão o supermercado.
 
\noindent Existem algumas regras que o horário deve cumprir, como por exemplo: 
\begin{itemize}
    \item Não podem existir menos de 3 funcionários a trabalhar por turno.
    \item Um funcionário só pode trabalhar 5 turnos.
    \item Um funcionário não pode trabalhar em dois turnos no mesmo dia.
    \item Todos os funcionários têm de ser colocados no horário.
\end{itemize}
 
 Assim sendo, se todas as regras acima definidas forem cumpridas, resta apenas que o posicionamento de funcionários pelo horário seja coerente mediante o número de clientes previstos para cada dia.
 
 \subsection*{\bf Estados}
 
 Como estado inicial do problema temos um horário vazio, horário este que é composto pelo dia da semana e com os turnos da manha e de tarde para cada dia, ao total temos 14 turnos diferentes. Este estado é definido por uma lista de dias (de segunda a domingo), para cada um destes dias estarão associados turnos (manha e tarde), estes turnos serão preenchidos com uma lista funcionários que trabalham no mesmo, como se trata do estado inicial as listas de funcionários estarão vazias.
 
 A cada novo estado posicionaríamos um funcionário nas listas de turnos escolhidos, ou seja, 5 turnos seriam preenchidos, o funcionário seria definido pelo seu número identificativo. Desta forma as listas seriam preenchidas com cada vez mais números de funcionários.
 
 No final da pesquisa ficaríamos com todos os funcionários inseridos em turnos pelo horário e não haveriam turnos vazios.
 \clearpage
 O formato utilizado para guardar os estados seria um dicionário, e teria uma formatação igual à seguinte:
 
 
\begin{lstlisting}
{
    "Segunda": {
        "manha":[], "tarde":[]              
    },
    "Terca":{
        "manha":[], "tarde":[]
    },
    "Quarta":{
        "manha":[], "tarde":[]
    },
    "Quinta":{
        "manha":[], "tarde":[]
    },
    "Sexta":{
        "manha":[], "tarde":[]
    },
    "Sabado":{
        "manha":[], "tarde":[]
    },
    "Domingo":{
        "manha":[], "tarde":[]
    }
}
\end{lstlisting}

\noindent Onde dentro de cada "manha" ou "tarde" estaria a lista de funcionário daquele dia.